\documentclass[a4paper,headings=small]{scrartcl}
\KOMAoptions{DIV=12}

\usepackage[utf8x]{inputenc}
\usepackage{amsmath}
\usepackage{graphicx}
\usepackage{multirow}
\usepackage{listings}
\usepackage{subfigure}

% define style of numbering
\numberwithin{equation}{section} % use separate numbering per section
\numberwithin{figure}{section}   % use separate numbering per section

% instead of using indents to denote a new paragraph, we add space before it
\setlength{\parindent}{0pt}
\setlength{\parskip}{10pt plus 1pt minus 1pt}

\title{Projective and direct Euclidean reconstruction}
\subtitle{Photogrammetric Computer Vision - WS12/13 - Excercise 6}
\author{\textbf{Team A}: Marcus Grum, Robin Vobruba, Marcus Pannwitz, Jens Jawer}
\date{\today}

\pdfinfo{%
  /Title    (Photogrammetric Computer Vision - WS12/13 - Excercise 6 - Projective and direct Euclidean reconstruction)
  /Author   (Team A: Marcus Grum, Robin Vobruba, Marcus Pannwitz, Jens Jawer)
  /Creator  ()
  /Producer ()
  /Subject  ()
  /Keywords ()
}

% Simple picture reference
%   Usage: \image{#1}{#2}{#3}
%     #1: file-name of the image
%     #2: percentual width (decimal)
%     #3: caption/description
%
%   Example:
%     \image{myPicture}{0.8}{My huge house}
%     See fig. \ref{fig:myPicture}.
\newcommand{\image}[3]{
	\begin{figure}[htbp]
		\centering
		\includegraphics[width=#2\textwidth]{#1}
		\caption{#3}
		\label{fig:#1}
	\end{figure}
}
\newcommand{\imgRootGenerated}{../../../target}


\begin{document}

\maketitle

\section{Results:}

In Fig. \ref{fig:\imgRootGenerated/projectiveReconstruction.png}), one can see the projective reconstruction
of Bach's head. Indeed, his silhouette is not visible at all.

\image{\imgRootGenerated/projectiveReconstruction.png}{0.2}{%
		Bach's head in 3d (Projective Reconstruction).}

In order to transform the projective reconstruction to an Euclidian reconstruction,
we used the projective points $X_P$ and Euclidian points $X_E$
to determine the spatial transformation $H$.

This has been applied to our object points $X_O$, which then have been visualized
in Fig. \ref{fig:\imgRootGenerated/euclidianReconstruction.png}).

\image{\imgRootGenerated/euclidianReconstruction.png}{0.2}{%
		Bach's head in 3d (Euclidian Reconstruction).}

Here, one can see Bach's head clearly.

\newpage
\section{Printed Code:}

\lstinputlisting[breaklines=true,language=C++]{../native/pcv6.cpp}

\end{document}

